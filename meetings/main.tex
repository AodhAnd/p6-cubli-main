%implementing ocument formatting:
\input{preamble.tex}
\begin{document}
\renewcommand\chaptername{KAPITEL}
\renewcommand\contentsname{Indhold}
\renewcommand\figurename{Figur}
\renewcommand\tablename{Tabel}

\section*{Supervisor meeting\\ \small Monday, 29th of February 2016}

\subsection{Generating Matlab Code (inputs/outputs)}
\begin{itemize}
  \item[-] Simon will answer these questions.
  \item[-] John will forward some Matlab code.
\end{itemize}

\subsection{Parameters for Simulation}
\begin{itemize}
  \item[-] The parameters are found on page 25 in the \si{AAU^3} report.
\end{itemize}

\subsection{Cubli Model}
\begin{itemize}
\item[-] Angle reference should be put on the drawings.
\item[-] \textbf{F} is not a torque but rather a translational force.
\item[-] Make complete correspondence between the two figures (concerning \si{\tau_w} and \nolinebreak\si{\tau_m})
\item[-] Vector notation - specify that the full equation is provided, however the system will only be addressed around the z-axis.
\item[-] The components of \textbf{F} could have been in a positive direction to match conventions.
\item[-] Equation 1.12 - for the dot followed immediately by a minus, at least use a parenthesis.
\item[-] Specify in equation 3.2 that \si{\ddot{\theta}_w} is the angular acceleration with respect to the frame.
\item[-] We write: "Vector F is composed of two linear [...]", write instead: "[...] decomposed into two forces parallel to the two axes."
\item[-] Where we write: "To further investigate [...]" \si{\rightarrow} The argument is instead that we want to put up Newtons 2nd law using [...] - too many words in what we write, be concrete.
\item[-] We write: "[...] composing the vector [...]". Write instead: "The vector is composed of [...]", or "constituting the vector".
\item[-] \si{\tau_M} should have been \si{\tau_m}.
\item[-] Vector cross product - scalar on left hand side and vector on the right hand side - can be fixed by using dot product.
\item[-] Verification of the model - Linear vs nonlinear - analysis of the model in time as well as in frequency domain.
\end{itemize}

\subsection{Next Supervisor meeting}
Monday, 7th of March at 13.30

\end{document}









