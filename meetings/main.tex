%implementing ocument formatting:
\input{preamble.tex}
\begin{document}
\renewcommand\chaptername{KAPITEL}
\renewcommand\contentsname{Indhold}
\renewcommand\figurename{Figur}
\renewcommand\tablename{Tabel}

\section*{Supervisor meeting\\ \small Monday, 7th of February 2016}

\subsection{Sent Material}
Corrective suggestions from the supervisors have been sent by email for later reference.

\subsection{Linearization}
\begin{itemize}
  \item[-] See the comments about the bar notation in the received correction pdfs.
  \item[-] Say that the $\theta_F$ and its derivative are 0 as well as the angle $\theta_w$. Then the torque ends up being zero.
  \item[-] About equation 3.19 to 3.21: Sum up what equations the model is made of and regroup the similar terms.
  \item[-] About figure 3.4: clarify what we want to show with this diagram. The `area' is more of an interval in which the linear approximation stays somewhat close to the real sine function.
\end{itemize}

\subsection{Verification of Model}
\begin{itemize}
  \item[-] The idea behind figure 3.6 is to compare the linearized equations and block diagram. It is still a good idea to use this graph of comparison between model and equations in the report.
  \item[-] It is also necessary to compare the original non-linear model with the linearized one.
\end{itemize}

\subsection{System Test (grouped data points)}
\begin{itemize}
  \item[-] Try to do a (rolling) average of the potentiometer data.
  \item[-] Show this process in the report.
\end{itemize}

\subsection{Miscelleanous}
\begin{itemize}
  \item[-] Motor model $\rightarrow$ use a simple model, matchig the behavior of the motor controller, at first and improve it later, once a working controller has been implemented.
  \item[-] Sensors $\rightarrow$ same thing: stay with the potentiometer for the moment
  \item[-] In later iterations, it might be possible to implement some filters for other sensor types.
\end{itemize}

\subsection{Next Supervisor Meeting}
Monday, 14th of March at 13.00

\end{document}


% TEGN
%-----------------------------
% Højrepil:		$\rightarrow$