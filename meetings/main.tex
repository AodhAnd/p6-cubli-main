%implementing ocument formatting:
\input{preamble.tex}
\begin{document}
\renewcommand\chaptername{KAPITEL}
\renewcommand\contentsname{Indhold}
\renewcommand\figurename{Figur}
\renewcommand\tablename{Tabel}

\section*{Supervisor meeting\\ \small Monday, 14th of March 2016}

\subsection{Report}
\begin{itemize}
  \item[-] Zoom in on the simulation where the approximation is relevant.
  \item[-] Bode stability margins builds on a system being stable, which can be inferred by Nyquist stability criteria.
  \item[-] Zoom on the left hand root locus, and explain that the right hand plot is a zoom at origin of the left plot.
  \item[-] Nyquist - the \si{Z_{RHP} = N + P_{RHP}} equation is not directly the Nyquist Stability Criterion, but you can infer stability from it.
  \item[-] L(S) instead of OL
  \item[-] Direction of encirclements = sign of \si{N} in equation: \si{Z_{RHP} = \pm N + P_{RHP}}
  \item[-] Root locus plot - use the SISOTOOL GUI to create the first controller as a first iteration.
\end{itemize}

\subsection{New Parameters}
\begin{itemize}
  \item[-] Identify the total moment of inertia and the distance to center of mass (least square estimation).
  \item[-] See testing... (damping)
\end{itemize}

\subsection{Testing}
\begin{itemize}
\item[-] Try to repeat the step test at 10 degrees in order to match initial conditions.
\item[-] We may have different values of the damping in simulation compared to the system.
\item[-] Slight difference in frequency might also have something to do with the damping.
\item[-] Parameter estimation could be done using least square.
\item[-] Put up discrete time model when using least square method.
\end{itemize}

\subsection{Date of Examination}
\begin{itemize}
\item[-] Can not promise anything, because of coordination with external censor.
\item[-] It will be attempted to place the examination around $\pm$20th of June.
\end{itemize}

\subsection{Next Supervisor meeting}
Monday, 21st of March at 13.00

\end{document}