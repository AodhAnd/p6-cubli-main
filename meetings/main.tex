%implementing document formatting:
%page setup (page size, text size, page layout, chapters start on a new page).
%memoir is a form of book class that supports any kind of document.
\documentclass[fleqn,a4paper,12pt,twoside,openany]{memoir}

%setting the header and footer in that order:
\setheadfoot{28pt}{28pt} %if any problems are encountered, try changing the latter 28pt with 1cm.

%general package syntax: \usepackage[options]{package}

%setting language:
\RequirePackage[danish, english]{babel}

\usepackage{siunitx}

%this package makes it possible to treat any element as a float,
%figures and tables are by default treated as floats.
%read http://en.wikibooks.org/wiki/LaTeX/Floats,_Figures_and_Captions to specify your float.
\usepackage{float}
\usepackage{wrapfig}
\usepackage{placeins}

%this package makes it possible to make theorems and examples:
\usepackage{amsthm}
%setting the style of examples (parameters: plain, definition, remark):
%(definition is usually used for examples)
\theoremstyle{definition}
%the frist parameter is the syntax used in the document, the second is that which is printed in LaTex.
\newtheorem{example}{Eksempel}

%making it possible to use æ, ø and å:
\usepackage[utf8]{inputenc}
%helps with word division when using æ, ø and å, and makes it ps-font rather than bmp:
\usepackage[T1]{fontenc}

%package for implementation of graphic files:
\usepackage{graphicx}

%package for captions
\usepackage[nooneline]{caption}

%%package for implementation of math:
\usepackage{amsmath , amsfonts , amssymb, float}

%allowing use of color:
\usepackage{color}
%allowing use of more colors also in tables (see: http://en.wikibooks.org/wiki/LaTeX/Colors):
\usepackage[usenames,dvipsnames,svgnames,table]{xcolor}

%hyperlinks in the tabel of contents - comment this out before the report is printed.
\usepackage{hyperref}
\hypersetup{
	bookmarks = true,  % Show 'bookmark'-frame in pdf.
	colorlinks = true, % True = colored links, False = framed links.
	citecolor = blue,  % Link color for references.
	linkcolor = blue,  % Link color in table of contents.
	urlcolor = blue,   % Link color for extern URLs.
}

%makes it possible to refer to the name of a chapter rather than just the number.
\usepackage{nameref}

%package for writing program code in latex
\usepackage{listings}

\lstset{ 
language=C,               	 	% choose the language of the code
basicstyle=\footnotesize,       % the size of the fonts that are used for the code
numbers=left,                   % where to put the line-numbers
numberstyle=\footnotesize,      % the size of the fonts that are used for the line-numbers
stepnumber=1,                   % the step between two line-numbers. If it is 1 each line will be numbered
numbersep=5pt,                  % how far the line-numbers are from the code
backgroundcolor=\color{white},  % choose the background color. You must add \usepackage{color}
showspaces=false,               % show spaces adding particular underscores
showstringspaces=false,         % underline spaces within strings
showtabs=false,                 % show tabs within strings adding particular underscores
frame=single,           		% adds a frame around the code
tabsize=2,          			% sets default tabsize to 2 spaces
captionpos=b,           		% sets the caption-position to bottom
breaklines=true,       			% sets automatic line breaking
breakatwhitespace=false,    	% sets if automatic breaks should only happen at whitespace
escapeinside={\%*}{*)}          % if you want to add a comment within your code
}

%setting references (using numbers) and supporting i.a. Chicargo-style:
\usepackage{etex}
\usepackage{etoolbox}
\usepackage{keyval}
\usepackage{ifthen}
\usepackage{url}
\usepackage{csquotes}
\usepackage[backend=biber,url=true,doi=true,style=numeric,sorting=none]{biblatex}
\bibliography{myBib.bib}

%this package makes it possible include pdf pages in fx appendix;
%using  following syntax: \includepdf[pages={1}]{myfile.pdf}
\usepackage{pdfpages}

%%%MARGINER%%%
\setlrmarginsandblock{3.5cm}{2.5cm}{*}	% \setlrmarginsandblock{inner margin}{outer margin}{ratio}
\setulmarginsandblock{2.5cm}{3.0cm}{*}	% \setulmarginsandblock{top}{bottom}{ratio}
\checkandfixthelayout 			            % fixes stuff..

%Enables the use FiXme refferences. Syntax: \fixme{...}
%With "final" in stead of "draft" an error will ocure for every FiXme
%under compilation.
\usepackage[footnote,draft,english,silent,nomargin]{fixme}

%%%CHAPTERLAYOUT%%%
%setting the color of the chapter number
\definecolor{numbercolor}{gray}{0.7}
%Downloaded chapter-setup:
\newif\ifchapternonum
\makechapterstyle{jenor}{
  \renewcommand\printchaptername{}
  \renewcommand\printchapternum{}
  \renewcommand\printchapternonum{\chapternonumtrue}
  \renewcommand\chaptitlefont{\fontfamily{pbk}\fontseries{db}\fontshape{n}\fontsize{25}{35}\selectfont\raggedleft}
  \renewcommand\chapnumfont{\fontfamily{pbk}\fontseries{m}\fontshape{n}\fontsize{1in}{0in}\selectfont\color{numbercolor}}
  \renewcommand\printchaptertitle[1]{%
    \noindent
    \ifchapternonum
    \begin{tabularx}{\textwidth}{X}
    {\let\\\newline\chaptitlefont ##1\par} 
    \end{tabularx}
    \par\vskip-2.5mm\hrule
    \else
    \begin{tabularx}{\textwidth}{Xl}
    {\parbox[b]{\linewidth}{\chaptitlefont ##1}} & \raisebox{-15pt}{\chapnumfont \thechapter}
    \end{tabularx}
    \par\vskip2mm\hrule
    \fi
  }
}
%setting chapter style:
\chapterstyle{jenor}

%depth of numbered headlines (part/chapter/section/subsection):
\setsecnumdepth{none}
\maxsecnumdepth{none}
%depth of the table of contents:
\settocdepth{section}

% Makes sure LaTeX does not stretch the text at page break:
\raggedbottom
%Vectors
\renewcommand{\vec}[1]{\boldsymbol{\mathbf{#1}}}
\begin{document}
\renewcommand\chaptername{KAPITEL}
\renewcommand\contentsname{Indhold}
\renewcommand\figurename{Figur}
\renewcommand\tablename{Tabel}

\section*{Supervisor meeting\\ \small Wednesday, 27th of April 2016}

\subsection{Parameter Estimation - Optimization}
\begin{itemize}
  \item[-] Remember to reference the previous project regarding parameters.
  \item[-] The steepest descent is using the gradient then a line search.
  \item[-] The Newton method is using the gradient and the Hessian then a line search (only suggested direction).
  \item[-] Squared (\si{^2}) should be the squared norm (inner product)
  \item[-] Gauss-Newton method is minimization of a vector function.
  \item[-] In equation 5.10 - correct the notation regarding vectors.
  \item[-] Remove left side in equation 5.2 and 5.4.
  \item[-] Explicitly write the algorithms used.
  \item[-] Gradient should not be a vector in figure 5.6 and 5.7 on page 27 (gradient is a scalar - give instead a line which leads to the point where the gradient method would make us go)
  \item[-] Reformulation: This expression can then be used to choose the value of \si{\vec{\delta}} such that THE APPROXIMATION OF \si{f(x)} is minimized.
  \item[-] 2nd derivative in 5.12 should not be P, but instead \si{y_m}.
  \item[-] In equation 5.11 and 5.12 write the derivatives reduced to \si{\partial\vec{\theta} ^2}.
  \item[-] Use a known parameter to test how close the estimation gets both with SensTool and our own implementation.
  \item[-] "Evident from figure 5.2 [...]". Make better/more description and list the parameters used.
  \item[-] Try to calculate the normed RMS error only until time 3 just to see the difference.
  \item[-] Write more clearly which is our own implementation between figure 5.11 and 5.12.
  \item[-] We can just use the line search implemented in matrix course, just remember source.
  \item[-] Gauss-Newton method, see slide 17-18 lecture 7 in matrix course.
\end{itemize}

\subsection{Controller Analysis Section}
\begin{itemize}
  \item[-] Try to calculate the closed loop poles of system with proportional controller. That is, root locus of closed loop with P-controller.
  \item[-] Root Locus of P-controller confirms the behavior in figure 7.1
  \item[-] In figure 7.2, correct figure text: It is not a Nyquist plot.
  \item[-] Block diagram in figure 7.3 is not saying much, however, we could add a K and D'(s) instead of D(s), and then use it to explain how K is scaled to generate the loci.
  \item[-] Root locus, figure 7.4, better described as using a proportional controller.
  \item[-] Reformulate conclusion on controller: "This means that another kind of controller in needed, which also takes care of the velocity of the wheel.", we should not imply that it is not possible.
  \item[-] Reason for pole left of zero is friction in motor.
  \item[-] Include the transfer function on pole/zero-form with parameters.
  \item[-] If the torque goes to zero then the velocity would decay because of the motor friction.
\end{itemize}

\subsection{State Space Controller}
\begin{itemize}
  \item[-] Working controller! :D
  \item[-] Continue investigating in frequency domain.
  \item[-] Focus on making feedback come from the sensors.
\end{itemize}

\subsection{Exam Date}
\begin{itemize}
  \item[-] Suggested dates: Friday 17th of June or Monday 20th of June
\end{itemize}

\subsection{Next Supervisor Meeting}
Wednesday, 4th of May at 13.00

\end{document}
