%implementing ocument formatting:
\input{preamble.tex}
\begin{document}
\renewcommand\chaptername{KAPITEL}
\renewcommand\contentsname{Indhold}
\renewcommand\figurename{Figur}
\renewcommand\tablename{Tabel}

\section*{Supervisor meeting\\ \small Thursday, 18th of December 2015}

\subsection{Study regulations}
Take these regulations as a whole: the general goal is to be able to:
\begin{itemize}
  \item[-] set a model
  \item[-] evaluate it
  \item[-] simulate it
  \item[-] analyze it
  \item[-] design a contoller with different methods (root locus, etc.)
  \item[-] ...
\end{itemize}

\subsection{Model}
\begin{itemize}
  \item[-] The rotational acceleration of the wheel should also account for the rest of the forces (torques) affecting the wheel.
  \item[-] The bearing of the wheel also applies some unpredictable force which affects the total movement of the wheel and can be drawn as we want.
  \item[-] Take a global coordinate system to have a good overview: it should be easier to combine the equations. When we reach the equations from the paper, we can also do it another way with respective (local) coordinate sytem and then use cross products to put back the two equations in the same coordinate system.
  \item[-] Do the model the "book way" :
    \begin{enumerate} 
      \item Free-body diagrams
      \item Put up all the forces applying on the system
      \item Describe it with equations
    \end{enumerate}
  \item[-] Also possible later if we have time: use Lagrange method using energies, it can be elegant.
\end{itemize}

\subsection{Sensors}
\begin{itemize}
\item[-] It is fine to try to use the IMUs, it should be possible to also use some parameters inside the code to run an observer and bypass the pot-meter.
\end{itemize}

\subsection{Code implementation process}
A lot of code to go through but we should stay on the top layer not to loose time.

\subsection{Parameters}
The approach of using the given parameters and verifying them later when we have made the controller up and running might be better since the point of this semester is the Control Engineering.

\subsection{Next Supervisor meeting}
Monday, 29th of December at 13.00

\end{document}


% TEGN
%-----------------------------
% Højrepil:		$\rightarrow$