%implementing document formatting:
\input{preamble.tex}
%Vectors
\renewcommand{\vec}[1]{\boldsymbol{\mathbf{#1}}}
\begin{document}
\renewcommand\chaptername{KAPITEL}
\renewcommand\contentsname{Indhold}
\renewcommand\figurename{Figur}
\renewcommand\tablename{Tabel}

\section*{Supervisor meeting\\ \small Friday, 13th of May 2016}

\subsection{Requirements}
\begin{itemize}
  \item[-] The base plate level requirement: Angle/inclination not level, and only within reasonable limits.
  \item[-] Remove requirement on duration of stable operation.
  \item[-] Correct spelling errors.
  \item[-] Theory for maximum catching angle with no initial velocity of the wheel should be presented with the test itself.
  \item[-] The theoretical angle (maximum angle with the limited current) is estimated not considering capabilities of the controller. Note this in the report.
\end{itemize}

\subsection{Complementary Filter}
\begin{itemize}
  \item[-] Be more precise on definition of x and y - they are the components of the linear acceleration measured by the accelerometer.
  \item[-] You should not have a figure with no explanation - figure caption should describe the figure fully, it does not matter if previous explanations are repeated.
  \item[-] Explain the offset of the accelerometer angle to frame angle (\si{\theta-\frac{\pi}{4}}).
  \item[-] "Equilibrium position is slightly different from the vertical one" - write a more precise statement: The center of mass is not the same as the geometrical center.
  \item[-] It is not absolutely clear how we are getting the data, this should be specified in the report main matter not only in the appendix (figure 9.2).
  \item[-] Put up an expression for the reading of the IMU as a function of the gravity and of the acceleration.
% \item[-] Control in frequency domain is required for the semester
  \item[-] Further frequency domain analysis: We can try to describe the transfer functions from input to each of the outputs. Can be done from state space description via matlab.
  \item[-] Describe the variables that appear in equations.
  \item[-] The gyro angle is the initial angle + the integration.
  \item[-] Estimation of drift could maybe be done through use of the accelerometer - include discussion on this.
  \item[-] Rewrite just above equation 9.6, and include the sample time.
  \item[-] Do not use the if and only if arrows! (double arrow)
  \item[-] Just after equation 9.8, explain that we are going from z-domain to discrete time domain.
  \item[-] Explain why we use the same cutoff frequencies for the two filters.
\end{itemize}

\subsection{Next Supervisor Meeting}
Monday, 23rd of May at 14.00

\end{document}