%implementing document formatting:
%page setup (page size, text size, page layout, chapters start on a new page).
%memoir is a form of book class that supports any kind of document.
\documentclass[fleqn,a4paper,12pt,twoside,openany]{memoir}

%setting the header and footer in that order:
\setheadfoot{28pt}{28pt} %if any problems are encountered, try changing the latter 28pt with 1cm.

%general package syntax: \usepackage[options]{package}

%setting language:
\RequirePackage[danish, english]{babel}

\usepackage{siunitx}

%this package makes it possible to treat any element as a float,
%figures and tables are by default treated as floats.
%read http://en.wikibooks.org/wiki/LaTeX/Floats,_Figures_and_Captions to specify your float.
\usepackage{float}
\usepackage{wrapfig}
\usepackage{placeins}

%this package makes it possible to make theorems and examples:
\usepackage{amsthm}
%setting the style of examples (parameters: plain, definition, remark):
%(definition is usually used for examples)
\theoremstyle{definition}
%the frist parameter is the syntax used in the document, the second is that which is printed in LaTex.
\newtheorem{example}{Eksempel}

%making it possible to use æ, ø and å:
\usepackage[utf8]{inputenc}
%helps with word division when using æ, ø and å, and makes it ps-font rather than bmp:
\usepackage[T1]{fontenc}

%package for implementation of graphic files:
\usepackage{graphicx}

%package for captions
\usepackage[nooneline]{caption}

%%package for implementation of math:
\usepackage{amsmath , amsfonts , amssymb, float}

%allowing use of color:
\usepackage{color}
%allowing use of more colors also in tables (see: http://en.wikibooks.org/wiki/LaTeX/Colors):
\usepackage[usenames,dvipsnames,svgnames,table]{xcolor}

%hyperlinks in the tabel of contents - comment this out before the report is printed.
\usepackage{hyperref}
\hypersetup{
	bookmarks = true,  % Show 'bookmark'-frame in pdf.
	colorlinks = true, % True = colored links, False = framed links.
	citecolor = blue,  % Link color for references.
	linkcolor = blue,  % Link color in table of contents.
	urlcolor = blue,   % Link color for extern URLs.
}

%makes it possible to refer to the name of a chapter rather than just the number.
\usepackage{nameref}

%package for writing program code in latex
\usepackage{listings}

\lstset{ 
language=C,               	 	% choose the language of the code
basicstyle=\footnotesize,       % the size of the fonts that are used for the code
numbers=left,                   % where to put the line-numbers
numberstyle=\footnotesize,      % the size of the fonts that are used for the line-numbers
stepnumber=1,                   % the step between two line-numbers. If it is 1 each line will be numbered
numbersep=5pt,                  % how far the line-numbers are from the code
backgroundcolor=\color{white},  % choose the background color. You must add \usepackage{color}
showspaces=false,               % show spaces adding particular underscores
showstringspaces=false,         % underline spaces within strings
showtabs=false,                 % show tabs within strings adding particular underscores
frame=single,           		% adds a frame around the code
tabsize=2,          			% sets default tabsize to 2 spaces
captionpos=b,           		% sets the caption-position to bottom
breaklines=true,       			% sets automatic line breaking
breakatwhitespace=false,    	% sets if automatic breaks should only happen at whitespace
escapeinside={\%*}{*)}          % if you want to add a comment within your code
}

%setting references (using numbers) and supporting i.a. Chicargo-style:
\usepackage{etex}
\usepackage{etoolbox}
\usepackage{keyval}
\usepackage{ifthen}
\usepackage{url}
\usepackage{csquotes}
\usepackage[backend=biber,url=true,doi=true,style=numeric,sorting=none]{biblatex}
\bibliography{myBib.bib}

%this package makes it possible include pdf pages in fx appendix;
%using  following syntax: \includepdf[pages={1}]{myfile.pdf}
\usepackage{pdfpages}

%%%MARGINER%%%
\setlrmarginsandblock{3.5cm}{2.5cm}{*}	% \setlrmarginsandblock{inner margin}{outer margin}{ratio}
\setulmarginsandblock{2.5cm}{3.0cm}{*}	% \setulmarginsandblock{top}{bottom}{ratio}
\checkandfixthelayout 			            % fixes stuff..

%Enables the use FiXme refferences. Syntax: \fixme{...}
%With "final" in stead of "draft" an error will ocure for every FiXme
%under compilation.
\usepackage[footnote,draft,english,silent,nomargin]{fixme}

%%%CHAPTERLAYOUT%%%
%setting the color of the chapter number
\definecolor{numbercolor}{gray}{0.7}
%Downloaded chapter-setup:
\newif\ifchapternonum
\makechapterstyle{jenor}{
  \renewcommand\printchaptername{}
  \renewcommand\printchapternum{}
  \renewcommand\printchapternonum{\chapternonumtrue}
  \renewcommand\chaptitlefont{\fontfamily{pbk}\fontseries{db}\fontshape{n}\fontsize{25}{35}\selectfont\raggedleft}
  \renewcommand\chapnumfont{\fontfamily{pbk}\fontseries{m}\fontshape{n}\fontsize{1in}{0in}\selectfont\color{numbercolor}}
  \renewcommand\printchaptertitle[1]{%
    \noindent
    \ifchapternonum
    \begin{tabularx}{\textwidth}{X}
    {\let\\\newline\chaptitlefont ##1\par} 
    \end{tabularx}
    \par\vskip-2.5mm\hrule
    \else
    \begin{tabularx}{\textwidth}{Xl}
    {\parbox[b]{\linewidth}{\chaptitlefont ##1}} & \raisebox{-15pt}{\chapnumfont \thechapter}
    \end{tabularx}
    \par\vskip2mm\hrule
    \fi
  }
}
%setting chapter style:
\chapterstyle{jenor}

%depth of numbered headlines (part/chapter/section/subsection):
\setsecnumdepth{none}
\maxsecnumdepth{none}
%depth of the table of contents:
\settocdepth{section}

% Makes sure LaTeX does not stretch the text at page break:
\raggedbottom
\begin{document}
\renewcommand\chaptername{KAPITEL}
\renewcommand\contentsname{Indhold}
\renewcommand\figurename{Figur}
\renewcommand\tablename{Tabel}

\section*{Supervisor meeting\\ \small Monday, 21st of March 2016}

\subsection{Report}
\begin{itemize}
  \item[-] Use consistent units in figures (if axis are in rad, don't use degrees in the caption)
  \item[-] "Impulse response" is rather an "Initial value response"
\end{itemize}

\subsection{Center of Mass Offset}
\begin{itemize}
  \item[-] For this offset, try to use a string with a weight attached to show that it exists
  \item[-] The offset we think we have found could be caused by cushion pads being non-symmetric, by the wheel being fixed off-center
  \item[-] We can take the Cubli apart to get the center of mass, by hanging it by 3 different corners
  \item[-] Later on, we can also estimate this center of mass again, using the known result as a sanity check
  \item[-] If anything else has to be done with the dismantled Cubli, it should be done now. The setup should not be taken apart again after.
  \item[-] Try to analyze the potentiometer readout, with angle as a reference.
\end{itemize}

\subsection{10 degrees Fall Test}
\begin{itemize}
  \item[-] It should be made very clear, why this test has been made.
  \item[-] It is up to us to decide where to put graphs and conclusions (appendix vs. report body)
  \item[-] It is fine to place comparison graphs in both places (test appendix + verification section)
  \item[-] Blue curve is made with non-linear simulation
  \item[-] Be careful when fixing the wheel in reality: it should also be taken into consideration when simulating
\end{itemize}

\subsection{Estimation of Parameters}
\begin{itemize}
  \item[-] The group is currently working on this estimation using `Senstools' toolbox, presented by Tom S. Pedersen in one of the Control Engineering lectures. 
  \item[-] When fixing the wheel, the corresponding blocks have to be taken out
\end{itemize}

\subsection{Controller Simulation}
\begin{itemize}
  \item[-] It is not necessarily a good idea to try to cancel an unstable pole, which an integrator is
  \item[-] The system should be strictly proper, i.e. it should have strictly more poles than zeros.
  \item[-] When transferring the controller from sisotool to simulink, recalculate the gain.
\end{itemize}

\subsection{Miscelleanous}
\begin{itemize}
  \item[-] The code has to be analyzed further on to check which part causes the potentiometer's weird behavior and be sure that the controllers can run with a reduced code.
  \item[-] The ADC does not cause any problem
  \item[-] We should now be able to implement our own controller on the Cubli, from scratch
  \item[-] Later on, the motor's model should be included in the control simulation to design the controller.
  \item[-] The motor controller is used in open-loop mode with a current reference input.
  \item[-] Make a test for the motor control, it may introduce a pole. Thereafter, one/two zeros more might be needed.
\end{itemize}

\subsection{Next Supervisor meeting}
Monday, 31st of March at 10.00

\end{document}