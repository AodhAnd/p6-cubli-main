%implementing document formatting:
\input{preamble.tex}
\begin{document}
\renewcommand\chaptername{KAPITEL}
\renewcommand\contentsname{Indhold}
\renewcommand\figurename{Figur}
\renewcommand\tablename{Tabel}

\section*{Supervisor meeting\\ \small Thursday, 31st of March 2016}

\subsection{Sense Tool Parameter Estimation}
\begin{itemize}
  \item[-] The slower damping of the model, is due to stiction.
  \item[-] When using SenseTool to estimate the parameters, make sure to describe the method used by the tool.
  \item[-] In general, use Newton method first then steepest descent.
\end{itemize}

\subsection{Potentiometer}
\begin{itemize}
  \item[-] Resolution of the potentiometer, 0.486 V over a range of 90.35 degrees.
  \item[-] Maybe there is a gain in the AD-converter - talk to Simon.
  \item[-] Maybe we can increase the voltage across the potentiometer.
  \item[-] Remember to document everything.
\end{itemize}

\subsection{Motor}
\begin{itemize}
  \item[-] The controller on the motor is a closed loop current controller (input: current reference (PWM)).
  \item[-] Measure how fast the closed loop is.
  \item[-] In the motor documentation the closed loop current controller sampling time is 53.6 kHz.
  \item[-] However the closed loop response time is what is interesting in this case.
  \item[-] We have to argue in the report - we have back EMF, which changes the current - how good is the loop at rejecting this?
  \item[-] There is probably a limit on the speed in the code since the motor can go really fast.
\end{itemize}

\subsection{Controller}
\begin{itemize}
  \item[-] Priority next is to start implementing the controller.
  \item[-] When describing the Root Locus design, argument how the loci are changed according to new pole/zero placements.
  \item[-] Also include information on general Root Locus design 
\end{itemize}

\subsection{Next Supervisor meeting}
Send material on Monday, 11th of April\\
Prepare 7 to 10 minutes of presentation each\\
Wednesday, 13th of April at 13.00

\end{document}