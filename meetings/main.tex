%implementing document formatting:
\input{preamble.tex}
%Vectors
\renewcommand{\vec}[1]{\boldsymbol{\mathbf{#1}}}
\begin{document}
\renewcommand\chaptername{KAPITEL}
\renewcommand\contentsname{Indhold}
\renewcommand\figurename{Figur}
\renewcommand\tablename{Tabel}

\section*{Supervisor meeting\\ \small Wednesday, 4th of May 2016}

\subsection{State Space}
\begin{itemize}
  \item[-] Choice of states - it should be mentioned that it is something we chose.
  \item[-] We can define the different outputs - can be done in different ways - in our case we define the outputs as what we want to control.
  \item[-] Write up how the \si{\vec{A}} and \si{\vec{B}} matrices are the derivatives of \si{f(\vec{x},\vec{u})} in the equilibrium position.
  \item[-] Controllability matrix should be \si{[\vec{B}\ \vec{AB}\ \vec{A}^2\vec{B}]} $\rightarrow$ nxn matrix for single input (\si{3x3} in our case). Observability and Controllability matrices are only allowed to have rank n.
  \item[-] Consider choosing the controller constants corresponding to the red graph.
  \item[-] Should be initial velocity not initial acceleration in description preceding figure 8.3.
  \item[-] Section on practical implementation of the controller - how to measure the three states needed for the controller.
  \item[-] Could it be implemented as just a feedback - just with the angle?
\end{itemize}

\subsection{Complementary Filter}
\begin{itemize}
  \item[-] It is working! :D
  \item[-] The complementary filter might not be able to take into account the low frequency accelerations added by movement of the frame, however because of the small magnitude of the disturbances in relation to gravity, it should not be a problem - look into this and document findings.
  \item[-] Placement of the IMU can be changed such that disturbances to the accelerometer are reduced.
  \item[-] Make difference equation for complementary filter.
  \item[-] Include comparison in frequency domain of the two filter implementations (the comparison could also be done in the time domain with step tests).
  \item[-] The complementary filter can be seen as some kind of rudimentary observer - an other option is to make a more general observer (next semester scope).
\end{itemize}

\subsection{Further Remarks}
\begin{itemize}
  \item[-] Make sure that it is easy to change between controller implementations.
  \item[-] Comment clearly in the code.
\end{itemize}

\subsection{Next Supervisor Meeting}
Friday, 13th of May (possibly at 10.00)\\
Time of meeting will be updated/confirmed by mail.

\end{document}