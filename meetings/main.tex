%implementing document formatting:
\input{preamble.tex}
\begin{document}
\renewcommand\chaptername{KAPITEL}
\renewcommand\contentsname{Indhold}
\renewcommand\figurename{Figur}
\renewcommand\tablename{Tabel}

\section*{Supervisor meeting\\ \small Monday, 21st of March 2016}

\subsection{Report}
\begin{itemize}
  \item[-] Use consistent units in figures (if axis are in rad, don't use degrees in the caption)
  \item[-] "Impulse response" is rather an "Initial value response"
\end{itemize}

\subsection{Center of Mass Offset}
\begin{itemize}
  \item[-] For this offset, try to use a string with a weight attached to show that it exists
  \item[-] The offset we think we have found could be caused by cushion pads being non-symmetric, by the wheel being fixed off-center
  \item[-] We can take the Cubli apart to get the center of mass, by hanging it by 3 different corners
  \item[-] Later on, we can also estimate this center of mass again, using the known result as a sanity check
  \item[-] If anything else has to be done with the dismantled Cubli, it should be done now. The setup should not be taken apart again after.
  \item[-] Try to analyze the potentiometer readout, with angle as a reference.
\end{itemize}

\subsection{10 degrees Fall Test}
\begin{itemize}
  \item[-] It should be made very clear, why this test has been made.
  \item[-] It is up to us to decide where to put graphs and conclusions (appendix vs. report body)
  \item[-] It is fine to place comparison graphs in both places (test appendix + verification section)
  \item[-] Blue curve is made with non-linear simulation
  \item[-] Be careful when fixing the wheel in reality: it should also be taken into consideration when simulating
\end{itemize}

\subsection{Estimation of Parameters}
\begin{itemize}
  \item[-] The group is currently working on this estimation using `Senstools' toolbox, presented by Tom S. Pedersen in one of the Control Engineering lectures. 
  \item[-] When fixing the wheel, the corresponding blocks have to be taken out
\end{itemize}

\subsection{Controller Simulation}
\begin{itemize}
  \item[-] It is not necessarily a good idea to try to cancel an unstable pole, which an integrator is
  \item[-] The system should be strictly proper, i.e. it should have strictly more poles than zeros.
  \item[-] When transferring the controller from sisotool to simulink, recalculate the gain.
\end{itemize}

\subsection{Miscelleanous}
\begin{itemize}
  \item[-] The code has to be analyzed further on to check which part causes the potentiometer's weird behavior and be sure that the controllers can run with a reduced code.
  \item[-] The ADC does not cause any problem
  \item[-] We should now be able to implement our own controller on the Cubli, from scratch
  \item[-] Later on, the motor's model should be included in the control simulation to design the controller.
  \item[-] The motor controller is used in open-loop mode with a current reference input.
  \item[-] Make a test for the motor control, it may introduce a pole. Thereafter, one/two zeros more might be needed.
\end{itemize}

\subsection{Next Supervisor meeting}
Monday, 31st of March at 10.00

\end{document}