%implementing document formatting:
\input{preamble.tex}
%Vectors
\renewcommand{\vec}[1]{\boldsymbol{\mathbf{#1}}}
\begin{document}
\renewcommand\chaptername{KAPITEL}
\renewcommand\contentsname{Indhold}
\renewcommand\figurename{Figur}
\renewcommand\tablename{Tabel}

\section*{Supervisor meeting\\ \small Wednesday, 27th of April 2016}

\subsection{Parameter Estimation - Optimization}
\begin{itemize}
  \item[-] Remember to reference the previous project regarding parameters.
  \item[-] The steepest descent is using the gradient then a line search.
  \item[-] The Newton method is using the gradient and the Hessian then a line search (only suggested direction).
  \item[-] Squared (\si{^2}) should be the squared norm (inner product)
  \item[-] Gauss-Newton method is minimization of a vector function.
  \item[-] In equation 5.10 - correct the notation regarding vectors.
  \item[-] Remove left side in equation 5.2 and 5.4.
  \item[-] Explicitly write the algorithms used.
  \item[-] Gradient should not be a vector in figure 5.6 and 5.7 on page 27 (gradient is a scalar - give instead a line which leads to the point where the gradient method would make us go)
  \item[-] Reformulation: This expression can then be used to choose the value of \si{\vec{\delta}} such that THE APPROXIMATION OF \si{f(x)} is minimized.
  \item[-] 2nd derivative in 5.12 should not be P, but instead \si{y_m}.
  \item[-] In equation 5.11 and 5.12 write the derivatives reduced to \si{\partial\vec{\theta} ^2}.
  \item[-] Use a known parameter to test how close the estimation gets both with SensTool and our own implementation.
  \item[-] "Evident from figure 5.2 [...]". Make better/more description and list the parameters used.
  \item[-] Try to calculate the normed RMS error only until time 3 just to see the difference.
  \item[-] Write more clearly which is our own implementation between figure 5.11 and 5.12.
  \item[-] We can just use the line search implemented in matrix course, just remember source.
  \item[-] Gauss-Newton method, see slide 17-18 lecture 7 in matrix course.
\end{itemize}

\subsection{Controller Analysis Section}
\begin{itemize}
  \item[-] Try to calculate the closed loop poles of system with proportional controller. That is, root locus of closed loop with P-controller.
  \item[-] Root Locus of P-controller confirms the behavior in figure 7.1
  \item[-] In figure 7.2, correct figure text: It is not a Nyquist plot.
  \item[-] Block diagram in figure 7.3 is not saying much, however, we could add a K and D'(s) instead of D(s), and then use it to explain how K is scaled to generate the loci.
  \item[-] Root locus, figure 7.4, better described as using a proportional controller.
  \item[-] Reformulate conclusion on controller: "This means that another kind of controller in needed, which also takes care of the velocity of the wheel.", we should not imply that it is not possible.
  \item[-] Reason for pole left of zero is friction in motor.
  \item[-] Include the transfer function on pole/zero-form with parameters.
  \item[-] If the torque goes to zero then the velocity would decay because of the motor friction.
\end{itemize}

\subsection{State Space Controller}
\begin{itemize}
  \item[-] Working controller! :D
  \item[-] Continue investigating in frequency domain.
  \item[-] Focus on making feedback come from the sensors.
\end{itemize}

\subsection{Exam Date}
\begin{itemize}
  \item[-] Suggested dates: Friday 17th of June or Monday 20th of June
\end{itemize}

\subsection{Next Supervisor Meeting}
Wednesday, 4th of May at 13.00

\end{document}
