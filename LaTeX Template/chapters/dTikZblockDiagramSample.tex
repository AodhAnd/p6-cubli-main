\begin{figure}[H]
  \begin{tikzpicture}[ auto,
                       thick,                         %<--setting line style
                       node distance=2cm,             %<--setting default node distance
                       scale=1.5,                     %<--|these two scale the whole thing
                       every node/.style={scale=1.5}, %<  |(always change both)
                       >=triangle 45 ]                %<--sets the arrowtype
    \draw%--------------------------------------------------------------------------------------------
    	%Drawing the Equation Blocks:
    	
    	%Naming the coordinate (0,0) input1, for later use:
    	node[shape=coordinate][](input1) at (0,0){}
  	
    	%node(sum1)                       ...creates a node with name sum1
    	%[sum, right of = input1]         ...create a sum-circle (defined in macros)
    	%                                    to the right of the input node
    	%{$\sum$}                         ...writes a summation symbol inside the circle
    	node(sum1) [sum, right of = input1] {$\sum$}
  	
    	%at (4.5,0) [block]               ...creates block (as defined in macros) at (4.5,0)
    	%{ $[...]$ }                      ...writes a transfer function in the block
    	node(transferfunction) at (4.8,0) [block] {\Large $\ \frac{a\cdot s + b}{c \cdot s^2 + d \cdot s + e}\ $ }
    
      %node(integrate) at (6.8,0)       ...creates a node with name integrate @ (6.8,0)
      %[block]                          ...creates a block (as defined in macros) @ specified node
      %{\Large $\frac{1}{s}$}           ...writes an integrator in the block
      node(integrate) at (7.8,0) [block] {\Large $\frac{1}{s}$}
    
      %node(feedBack)                  ...creates node named feedBack
      %[block, below of = block1]      ...creates block below block1
      %{\Large$H(s)$}                  ...writes H(s) in the feed back block
      node(feedBack) [block, below of = transferfunction] {\large$H(s)$}
      
    ;%------------------------------------------------------------------------------------------------
    
    %Joining the Blocks
    %When using \draw with options like [->], each line must be enclosed as follows:
    %\draw[..arrow/line..]  ..handelingOfArrow/Line..  ;
    
    %Naming the coordinate (9,0) output1:
    \draw
      node[shape=coordinate][](output1) at (9,0){}
    ;
    
    % \draw[->](input1) --            ...draws a straight arrow from node with name: input1
    % node {$X(Z)$}(sum1)             ...to the node with name: sum1 .. writes on arrow: X(s)
  	\draw[->](input1) -- node {$X(s)$}(sum1);
  	
  	%same logic for the following as for the previous line
    \draw[->](sum1) -- node {} (transferfunction);
  	\draw[->](transferfunction) -- node {} (integrate);
  	
  	%\draw[->](feedBack)            ...draws an arrow from feedBack block
  	%-| node{} (sum1)               ...makes a 90 degree turn and terminates @ node: sum1
  	\draw[->](feedBack) -| node{} (sum1);

  	%\draw[->] (integrate)          ...draws straight arrow from integrate
  	%-- (output1)                   ...to (output1)
  	%|-                             ...making a 90 degree turn @ (output1)
  	%(feedBack)                     ...connects the arrow from (output1) to node: feedback
  	\draw[-] (integrate) -- (output1);
  	\draw[->] (output1) |- (feedBack);
  	
  	%Drawing output arrow from (output1) to (10.5,0)
  	\draw[->](output1) -- node {$Y(s)$} (10.5,0);

    %Drawing node(s) with \textbullet
    %[shift={... adjusts the position of the bullets
    \draw%--------------------------------------------------------------
      node at (input1)  [shift={(-0.04, -0.05 )}] {\Large \textopenbullet}
    	node at (output1) [shift={( 0.01, -0.03 )}] {\textbullet}
    ;%------------------------------------------------------------------

    %Drawing + and - @ sum-block
    \draw%--------------------------------------------------------------
      node at (sum1) [right = -6.5mm, below = .6mm] {$+$} %<--Plus
      node at (sum1) [right = -3mm, below = 3.9mm]  {$-$} %<--Minus
    ;%------------------------------------------------------------------

  \end{tikzpicture}
  \caption{Some tikzpicture drawing}
\end{figure}