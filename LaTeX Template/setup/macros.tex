%%%%%%%%%%%%%%%%%%%%%%%%%%%%%%%%%%%%%%%%%%%%%%%%%%%%%
%             UNITS, EQUATIONS AND TEXT             %
%%%%%%%%%%%%%%%%%%%%%%%%%%%%%%%%%%%%%%%%%%%%%%%%%%%%%
%Units:
\newcommand{\unit}[1]{&& \left[\si{#1}\right]} %\newcommand{\unit}[1]{[\si{#1}]}             <<| Use these if you want equations to be
\newcommand{\unitWh}[1]{[\si{#1}]}             %\newcommand{\eq}[2]{&&\si{#1} &= \si{#2}&&}  <<| centered.. .. will appear scrambled
\newcommand{\numUnit}[1]{\ \si{#1}&}           %                                               | from one equation to the next though..
%Equation:                                     %                                               | and does not work with long equations.. :/
\newcommand{\eq}[2]{\si{#1} &= \si{#2}}
\newcommand{\arw}{&& &\Updownarrow&&}
\newcommand{\eqOne}[2]{\si{#1} &= \si{#2} &\nonumber\\}
\newcommand{\eqTwo}[1]{&\ \ \ \ \si{#1}&}
%Text:
\newcommand{\tx}[1]{\text{#1}}
%Vectors
\renewcommand{\vec}[1]{\boldsymbol{\mathbf{#1}}}
%Vertical line in equations ie. |_x=y (whereTwo stacks two equalities at the line)
\newcommand{\where}[1]{ \left.\rule{0cm}{.5cm}\right\vert\rule{0cm}{.4cm}_{\substack{\rule{0cm}{.15cm}\\ \si{#1} }} }
\newcommand{\whereTwo}[2]{ \left.\rule{0cm}{.67cm}\right\vert\rule{0cm}{.5cm}_{\substack{\si{#1} \rule{0cm}{.19cm}\\\vspace{-.1cm}\\ \si{#2}}} }

%%%%%%%%%%%%%%%%%%%%%%%%%%%%%%%%%%%%%%%%%%%%%%%%%%%%%
%                 TIKZ SETTINGS                     %
%%%%%%%%%%%%%%%%%%%%%%%%%%%%%%%%%%%%%%%%%%%%%%%%%%%%%
\tikzset{
  block/.style    = {draw, thick, rectangle,
                     minimum height = 3em,
                     minimum width = 3em},
  sum/.style      = {draw, circle}, % Adder
}

%%%%%%%%%%%%%%%%%%%%%%%%%%%%%%%%%%%%%%%%%%%%%%%%%%%%%
%                  REFERENCES                       %
%%%%%%%%%%%%%%%%%%%%%%%%%%%%%%%%%%%%%%%%%%%%%%%%%%%%%

%Chapter
\newcommand{\Chapref}[1]{\emph{Chapter \ref{#1}}}
\newcommand{\chapref}[1]{\emph{chapter \ref{#1}}}
%Section
\newcommand{\Secref}[1]{\emph{Section \ref{#1}}}
\newcommand{\secref}[1]{\emph{section \ref{#1}}}
%subSection
\newcommand{\Subsecref}[1]{\emph{Subsection \ref{#1}}}
\newcommand{\subsecref}[1]{\emph{subsection \ref{#1}}}
%Appendix
\newcommand{\Appref}[1]{\emph{Appendix \ref{#1}}}
\newcommand{\appref}[1]{\emph{appendix \ref{#1}}}
%Listings
\newcommand{\Coderef}[1]{\emph{Listings: \ref{#1}}}
\newcommand{\coderef}[1]{\emph{listings: \ref{#1}}}
%Figure:
\newcommand{\Figref}[1]{\emph{Figure \ref{#1}}}
\newcommand{\figref}[1]{\emph{figure \ref{#1}}}
%Table:
\newcommand{\Tableref}[1]{\emph{Table \ref{#1}}}
\newcommand{\tableref}[1]{\emph{table \ref{#1}}}

%Expressions:
\newcommand{\Expr}[1]{\emph{Expression (\ref{#1})}}
\newcommand{\expr}[1]{\emph{expression (\ref{#1})}}

%Equations:
%1 equation:
\newcommand{\Eqref}[1]{\emph{Equation (\ref{#1})}}
\renewcommand{\eqref}[1]{\emph{equation (\ref{#1})}}
%2 equations:
\newcommand{\EqrefTwo}[2]{\emph{Equation (\ref{#1})} and \emph{(\ref{#2})}}
\newcommand{\eqrefTwo}[2]{\emph{equation (\ref{#1})} and \emph{(\ref{#2})}}
%3 equations:
\newcommand{\EqrefThree}[3]{\emph{Equation (\ref{#1})}, \emph{(\ref{#2})} and \emph{(\ref{#3})}}
\newcommand{\eqrefThree}[3]{\emph{equation (\ref{#1})}, \emph{(\ref{#2})} and \emph{(\ref{#3})}}
%4 equations:
\newcommand{\EqrefFour}[4]{\emph{Equation (\ref{#1})}, \emph{(\ref{#2})}, \emph{(\ref{#3})} and \emph{(\ref{#4})}}
\newcommand{\eqrefFour}[4]{\emph{equation (\ref{#1})}, \emph{(\ref{#2})}, \emph{(\ref{#3})} and \emph{(\ref{#4})}}
%5 equations:
\newcommand{\EqrefFive}[5]{\emph{Equation (\ref{#1})}, \emph{(\ref{#2})}, \emph{(\ref{#3})}, \emph{(\ref{#4})} and \emph{(\ref{#5})}}
\newcommand{\eqrefFive}[5]{\emph{equation (\ref{#1})}, \emph{(\ref{#2})}, \emph{(\ref{#3})}, \emph{(\ref{#4})} and \emph{(\ref{#5})}}
%6 equations:
\newcommand{\EqrefSix}[6]{\emph{Equation (\ref{#1})}, \emph{(\ref{#2})}, \emph{(\ref{#3})}, \emph{(\ref{#4})}, \emph{(\ref{#5})} and \emph{(\ref{#6})}}
\newcommand{\eqrefSix}[6]{\emph{equation (\ref{#1})}, \emph{(\ref{#2})}, \emph{(\ref{#3})}, \emph{(\ref{#4})}, \emph{(\ref{#5})} and \emph{(\ref{#6})}}
%7 equations:
\newcommand{\EqrefSeven}[7]{\emph{Equation (\ref{#1})}, \emph{(\ref{#2})}, \emph{(\ref{#3})}, \emph{(\ref{#4})}, \emph{(\ref{#5})}, \emph{(\ref{#6})} and \emph{(\ref{#7})}}
\newcommand{\eqrefSeven}[7]{\emph{equation (\ref{#1})}, \emph{(\ref{#2})}, \emph{(\ref{#3})}, \emph{(\ref{#4})}, \emph{(\ref{#5})}, \emph{(\ref{#6})} and \emph{(\ref{#7})}}